%        File: annealing.tex
%     Created: Mon Jun 21 04:00 PM 2021 I
% Last Change: Mon Jun 21 04:00 PM 2021 I
%
\documentclass[a4paper]{report}
\begin{document}
The optimisations described in this report have been implemented on a quantum computer using the resources provided by DWave Systems.

\section{Problem Description}
Given $N$ stocks, their covariance matrix and average returns over a time period (computable using classical computers too since it can be done beforehand, or offline),
an (optional) expected average return amount, (optional) desired number of stocks, to come up with a portfolio that satisfies
these conditions to different degrees and also minimises the investment risk.

\section{Boolean Variables}
Define a boolean variable $x_i$ for the $i^{th}$ stock. $x_i = 1$ means that we include this stock in our portfolio, 
$x_i = 0$ means that we do not include this stock in our portfolio.

\section{Objective Function}
The objective function we want to minimise is the investment risk $R$ incurred. It is defined as follows-
$$ R = \sum\limits_{i = 1}^{n}\sigma_{ii}x_i^2 + \sum\limits_{i = 1}^{n}\sum\limits_{j = i + 1}^{n}\sigma_{ij}x_ix_j$$
where $\sigma_{ij}$ is the covariance between the $i^{th}$ and the $j^{th}$ stocks. ($\sigma_{ii}$ is the variance of
the $i^{th}$ stock.)
\section{Fixed number of Stocks Constraint}
We want to allow exactly $f$ stocks, hence the following constraint-
$$(\sum\limits_{i = 1}^{n} x_i - f)^2 = 0$$
This gives a quadratic form equation. To satisfy this, we need exactly $f$ many $x_i$ set to $1$.
\section{Desired Return Constraint}
We want to ensure that the Returns we get are as close to the Desired Returns as possible, therefore-
$$(\sum\limits_{i = 1}^{n} r_i x_i - \mu_p)^2 = 0$$
where $r_i$ denotes the average returns of the $i^{th}$ stock and $\mu_p$ is the Desired Returns.
\end{document}


